\documentclass[a4paper,12pt]{book}
\usepackage{fullpage}
\usepackage{setspace}
\usepackage{multicol}
\usepackage{amsthm}
\usepackage{tikz}
\newcommand*\circled[1]{\tikz[baseline=(char.base)]{
            \node[shape=circle,draw,inner sep=2pt] (char) {#1};}}
\pagestyle{empty}

\theoremstyle{defn}
\newtheorem{defn}{Definition}[section]
\theoremstyle{expl}
\newtheorem{expl}{Example}[section]

\begin{document} 


\chapter{Linear Equations in Linear Algebra}
\section{Systems of Linear Equations}



\begin{defn}\textup{A Linear Equations is the variables $x_1,x_2...x_n$ is an equation that can be written in the form $a_1x_1+a_2x_2+...+a_nx_n = b$ where $a_1,a_2, ..., a_n$ are real coefficient and $b$ is a real number(and known)} \end{defn}
\begin{defn}\textup{A System of Linear Equations
$\left\{ \begin{array}{c}
a_{11}x_1+a_{12}x_2+...+a_{1n}x_n = b_1\\
a_{21}x_1+a_{22}x_2+...+a_{2n}x_n = b_2\\
\vdots\\
a_{m1}x_1+a_{m2}x_2+...+a_{mn}x_n = b_m
\end{array} \right.$ m number of equations, n number of unknowns (standard form) (first index row number, second index col number)}\end{defn}
\begin{defn}\textup{A solution of the system is a list ($s_1,s_2,...,s_n$) of numbers that makes each equation a true statment when the values are substituted for $x_1,x_2,...,x_n$}\end{defn}
\begin{defn}\textup{Solution Set is the set of all possible solutions}\end{defn}

Geometric Interpretations
Example) Find the Solution set of the system \\
(a) $\left\{ \begin{array}{rcl}
x_1 - x_2 & = & 5 \\
2x_1 + x_2 & = & 7
\end{array} \right.$\\
(b) $\left\{ \begin{array}{rcl}
x_1 - 2x_2&=& 4 \\
-2x_1 + 4x_2 &=& -8
\end{array} \right.$\\
(c) $\left\{ \begin{array}{rcl}
x_1+3x_2&=&1\\
2x_1+6x_2&=&5
\end{array} \right.$


\begin{defn}\textup{A linear system is consistant if it has either one solution or infinitely many solutions}\end{defn}

\begin{defn}\textup{Matrix of Coefficients} 
$\left[ \begin{array}{cccc}
a_{11} & a_{12} & \ldots & a_{1n}\\
a_{21} & a_{22} & \ldots & a_{2n}\\
\vdots &\vdots&\ddots& \vdots\\
a_{m1} & a_{m2} & \ldots & a_{mn}\\
\end{array} \right]$ \end{defn}


\begin{defn}\textup{Augmented Matrix of the System}
$\left[ \begin{array}{cccc|c}
a_{11} & a_{12} & \ldots & a_{1n} & b_1\\
a_{21} & a_{22} & \ldots & a_{2n} & b_2\\
\vdots &\vdots&\ddots& \vdots &\vdots\\
a_{m1} & a_{m2} & \ldots & a_{mn} & b_m\\
\end{array}\right]$ \end{defn}


\section{Row Reduction and Echelon Forms}
\begin{defn}\textup{A leading  of a row in a matrix is the left most non-zero entry} \end{defn}
Example) $\left[ \begin{array}{cccccc} 0 & 0 & \circled{7} & 3 & 4 & 1\\ \circled{2} & 4 & 0 &0 &0 &0 \\ 0 &0 &0 &0 &\circled{-2} &0 \end{array} \right] $

\begin{defn}\textup{A rectangular matrix is in echelon form if it has the following three properties:
\begin{enumerate}
\item All non-zero rows are above any zero rows.
\item Each leading entry of a row is in a column to the right of the leading entry above it.
\item All entries in a column below a leading entry are zero.
\end{enumerate}} \end{defn}

\section{Vector Equations}
\begin{defn}
\textup{Vectors\\
In $R^2, \vec{v} = \left[\begin{array}{c} v_1 \\ v_2 \end{array}\right]$, in $R^3, \vec{v}=\left[\begin{array}{c} v_1 \\ v_2 \\v_3\end{array}\right]$, in $R^n, \vec{v} =\left[\begin{array}{c} v_1 \\ v_2 \\ \vdots \\ v_n\end{array}\right]$}\end{defn}
\begin{defn}
\textup{
Alebraic Operations of Vectors.\\
$\vec{u}=\left[\begin{array}{c} u_1\\u_2\\\vdots\\u_n \end{array}\right]$
$\vec{v}=\left[\begin{array}{c} v_1\\v_2\\\vdots\\v_n \end{array}\right]$\\
Addition: $\vec{u}+\vec{v}= \left[\begin{array}{c} u_1+v_1\\u_2+v_2\\\vdots\\u_n+v_n \end{array}\right]$\\
Multipy by Scaler: $c\in R\quad c\vec{v}=\left[\begin{array}{c} cv_1\\cv_2\\\vdots\\cv_n \end{array}\right]$
}\end{defn}

\begin{defn}
\textup{
Linear Combination of Vectors\\
$\vec{v_1},\vec{v_2},...,\vec{v_p}$ vectors in $R^n$\\
$c_1,c_2,...,c_n$ scalers\\
Linear Combination: $c_1\vec{v_1}+c_2\vec{v_2}+...+c_n\vec{v_n}$
}\end{defn}

\begin{defn}
\textup{
Vector Form of a System of Linear Equations\\
$\begin{array}{ccccccccc}
a_{11}x_1&+&a_{12}x_2&+&...&+&a_{1n}x_n &=&b_1\\
a_{21}x_1&+&a_{22}x_2&+&...&+&a_{2n}x_n &=&b_2\\
\vdots&&\vdots\\
a_{m1}x_1&+&a_{m2}x_2&+&...&+&a_{mn}x_n&=& b_m\\
\vec{a_1}&&\vec{a_2}&&&&\vec{a_n}&&\vec{b_n}\\
\end{array}$\\
Short Vector Form: $\vec{a_1}x_1+\vec{a_2}x_2+...+\vec{a_n}x_n=\vec{b_n}$\\
Long Vector Form: $\left[\begin{array}{c} a_{11}\\a_{21}\\\vdots\\a_{m1}\end{array}\right]x_1
+\left[\begin{array}{c} a_{12}\\a_{22}\\\vdots\\a_{m2}\end{array}\right]x_2+...
+\left[\begin{array}{c} a_{1n}\\a_{2n}\\\vdots\\a_{mn}\end{array}\right]x_n
=\left[\begin{array}{c} b_{1}\\b_{2}\\\vdots\\b_{m}\end{array}\right]$
}\end{defn}
\begin{expl}\-\\
\textup{
Standard Form: $\left\{ \begin{array}{ccccccc}
2x_1&+&3x_2&-&4x_3&=&5\\
x_1& & &+& 2x_3&=&1\\
 & & x_2 & - &x_3&=&4
\end{array}
\right.$\\
Augmented Matrix:
$\left[ \begin{array}{ccc|c}
2&3&-4&5\\
1&0&2&1\\
0&1&-1&4
\end{array}
\right]$\\
Vector Form:
$\left[ \begin{array}{c} 2\\1\\0 \end{array} \right]x_1+
\left[ \begin{array}{c} 3\\0\\1 \end{array} \right]x_2+
\left[ \begin{array}{c} -4\\2\\-1 \end{array} \right]x_3=
\left[ \begin{array}{c} 5\\1\\4 \end{array} \right]$
}\end{expl}
\begin{defn}\-\\
\textup{
If $\vec{v_1},\vec{v_2},...,\vec{v_p}$ are vectors in $R^N$ then the set of all linear combonations of $\vec{v_1},\vec{v_2},...,\vec{v_p}$ is denoted by Span\{$\vec{v_1},\vec{v_2},...,\vec{v_p}$\} and is called a subset of $R^N$ spanned (or generated) by $\vec{v_1},\vec{v_2},...,\vec{v_p}$.
}\end{defn}
\begin{expl}\-\\
\textup{
for $R^3$, describe Span\{$\left[\begin{array}{c}1\\0\\0\end{array}\right]$,$\left[\begin{array}{c}0\\0\\1\end{array}\right]$\}\\
All Linear Combontations we get the $x_1x_3$-plane\\
}\end{expl}
Remark: A system of linear equations is consistant if $\vec{b}$ is in Span\{$\vec{a_1},\vec{a_2},...,\vec{a_n}$ \}\\
\begin{expl}\-\\
\textup{
Determine if $\vec{b} = \left[\begin{array}{c}11\\-5\\9\end{array}\right]$ is in the Span\{$\left[\begin{array}{c}1\\0\\1\end{array}\right],
\left[\begin{array}{c}-2\\1\\2\end{array}\right],
\left[\begin{array}{c}-6\\7\\5\end{array}\right]$ \}\\
}\end{expl}

\section{The Matrix Equation $A\vec{x}=\vec{b}$}
Standard Form: 
$\begin{array}{ccccccccc}
a_{11}x_1&+&a_{12}x_2&+&...&+&a_{1n}x_n &=&b_1\\
a_{21}x_1&+&a_{22}x_2&+&...&+&a_{2n}x_n &=&b_2\\
\vdots&&\vdots&&&&\vdots&&\vdots\\
a_{m1}x_1&+&a_{m2}x_2&+&...&+&a_{mn}x_n&=& b_m\\
\end{array}$\\
\-\\ Matrix Form: 
$\left[ \begin{array}{cccc}
a_{11} & a_{12} & \ldots & a_{1n}\\
a_{21} & a_{22} & \ldots & a_{2n}\\
\vdots &\vdots&\ddots& \vdots\\
a_{m1} & a_{m2} & \ldots & a_{mn}\\
\end{array} \right]
\left[\begin{array}{c}x_1\\x_2\\\vdots\\x_m\end{array}\right]=
\left[\begin{array}{c}b_1\\b_2\\\vdots\\b_m\end{array}\right]$\\
Theorem: The system $A\vec{x}=\vec{b}$ has a solution Iff $\vec{b}$ is a linear combination of $A, \vec{b} \in$ Span\{column vectors of $A$\}

\end{document}