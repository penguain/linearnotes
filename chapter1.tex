\documentclass[a4paper,12pt]{book}
\usepackage{fullpage}
\usepackage{setspace}
\usepackage{multicol}
\usepackage{amsthm}
\usepackage{amssymb}
\usepackage{tikz}
\newcommand*\circled[1]{\tikz[baseline=(char.base)]{
            \node[shape=circle,draw,inner sep=2pt] (char) {#1};}}
\pagestyle{empty}

\theoremstyle{defn}
\newtheorem{defn}{Definition}[section]
\theoremstyle{expl}
\newtheorem{expl}{Example}[section]

\begin{document} 


\chapter{Linear Equations in Linear Algebra}
\section{Systems of Linear Equations}



\begin{defn}\textup{A Linear Equations is the variables $x_1,x_2...x_n$ is an equation that can be written in the form $a_1x_1+a_2x_2+...+a_nx_n = b$ where $a_1,a_2, ..., a_n$ are real coefficient and $b$ is a real number(and known)} \end{defn}
\begin{defn}\textup{A System of Linear Equations
$\left\{ \begin{array}{c}
a_{11}x_1+a_{12}x_2+...+a_{1n}x_n = b_1\\
a_{21}x_1+a_{22}x_2+...+a_{2n}x_n = b_2\\
\vdots\\
a_{m1}x_1+a_{m2}x_2+...+a_{mn}x_n = b_m
\end{array} \right.$ m number of equations, n number of unknowns (standard form) (first index row number, second index col number)}\end{defn}
\begin{defn}\textup{A solution of the system is a list ($s_1,s_2,...,s_n$) of numbers that makes each equation a true statment when the values are substituted for $x_1,x_2,...,x_n$}\end{defn}
\begin{defn}\textup{Solution Set is the set of all possible solutions}\end{defn}

Geometric Interpretations
Example) Find the Solution set of the system \\
(a) $\left\{ \begin{array}{rcl}
x_1 - x_2 & = & 5 \\
2x_1 + x_2 & = & 7
\end{array} \right.$\\
(b) $\left\{ \begin{array}{rcl}
x_1 - 2x_2&=& 4 \\
-2x_1 + 4x_2 &=& -8
\end{array} \right.$\\
(c) $\left\{ \begin{array}{rcl}
x_1+3x_2&=&1\\
2x_1+6x_2&=&5
\end{array} \right.$


\begin{defn}\textup{A linear system is consistant if it has either one solution or infinitely many solutions}\end{defn}

\begin{defn}\textup{Matrix of Coefficients} 
$\left[ \begin{array}{cccc}
a_{11} & a_{12} & \ldots & a_{1n}\\
a_{21} & a_{22} & \ldots & a_{2n}\\
\vdots &\vdots&\ddots& \vdots\\
a_{m1} & a_{m2} & \ldots & a_{mn}\\
\end{array} \right]$ \end{defn}


\begin{defn}\textup{Augmented Matrix of the System}
$\left[ \begin{array}{cccc|c}
a_{11} & a_{12} & \ldots & a_{1n} & b_1\\
a_{21} & a_{22} & \ldots & a_{2n} & b_2\\
\vdots &\vdots&\ddots& \vdots &\vdots\\
a_{m1} & a_{m2} & \ldots & a_{mn} & b_m\\
\end{array}\right]$ \end{defn}


\section{Row Reduction and Echelon Forms}
\begin{defn}\textup{A leading  of a row in a matrix is the left most non-zero entry} \end{defn}
Example) $\left[ \begin{array}{cccccc} 0 & 0 & \circled{7} & 3 & 4 & 1\\ \circled{2} & 4 & 0 &0 &0 &0 \\ 0 &0 &0 &0 &\circled{-2} &0 \end{array} \right] $

\begin{defn}\textup{A rectangular matrix is in echelon form if it has the following three properties:
\begin{enumerate}
\item All non-zero rows are above any zero rows.
\item Each leading entry of a row is in a column to the right of the leading entry above it.
\item All entries in a column below a leading entry are zero.
\end{enumerate}} \end{defn}

\section{Vector Equations}
\begin{defn}
\textup{Vectors\\
In $R^2, \vec{v} = \left[\begin{array}{c} v_1 \\ v_2 \end{array}\right]$, in $R^3, \vec{v}=\left[\begin{array}{c} v_1 \\ v_2 \\v_3\end{array}\right]$, in $R^n, \vec{v} =\left[\begin{array}{c} v_1 \\ v_2 \\ \vdots \\ v_n\end{array}\right]$}\end{defn}
\begin{defn}
\textup{
Alebraic Operations of Vectors.\\
$\vec{u}=\left[\begin{array}{c} u_1\\u_2\\\vdots\\u_n \end{array}\right]$
$\vec{v}=\left[\begin{array}{c} v_1\\v_2\\\vdots\\v_n \end{array}\right]$\\
Addition: $\vec{u}+\vec{v}= \left[\begin{array}{c} u_1+v_1\\u_2+v_2\\\vdots\\u_n+v_n \end{array}\right]$\\
Multipy by Scaler: $c\in R\quad c\vec{v}=\left[\begin{array}{c} cv_1\\cv_2\\\vdots\\cv_n \end{array}\right]$
}\end{defn}

\begin{defn}
\textup{
Linear Combination of Vectors\\
$\vec{v_1},\vec{v_2},...,\vec{v_p}$ vectors in $R^n$\\
$c_1,c_2,...,c_n$ scalers\\
Linear Combination: $c_1\vec{v_1}+c_2\vec{v_2}+...+c_n\vec{v_n}$
}\end{defn}

\begin{defn}
\textup{
Vector Form of a System of Linear Equations\\
$\begin{array}{ccccccccc}
a_{11}x_1&+&a_{12}x_2&+&...&+&a_{1n}x_n &=&b_1\\
a_{21}x_1&+&a_{22}x_2&+&...&+&a_{2n}x_n &=&b_2\\
\vdots&&\vdots\\
a_{m1}x_1&+&a_{m2}x_2&+&...&+&a_{mn}x_n&=& b_m\\
\vec{a_1}&&\vec{a_2}&&&&\vec{a_n}&&\vec{b_n}\\
\end{array}$\\
Short Vector Form: $\vec{a_1}x_1+\vec{a_2}x_2+...+\vec{a_n}x_n=\vec{b_n}$\\
Long Vector Form: $\left[\begin{array}{c} a_{11}\\a_{21}\\\vdots\\a_{m1}\end{array}\right]x_1
+\left[\begin{array}{c} a_{12}\\a_{22}\\\vdots\\a_{m2}\end{array}\right]x_2+...
+\left[\begin{array}{c} a_{1n}\\a_{2n}\\\vdots\\a_{mn}\end{array}\right]x_n
=\left[\begin{array}{c} b_{1}\\b_{2}\\\vdots\\b_{m}\end{array}\right]$
}\end{defn}
\begin{expl}\-\\
\textup{
Standard Form: $\left\{ \begin{array}{ccccccc}
2x_1&+&3x_2&-&4x_3&=&5\\
x_1& & &+& 2x_3&=&1\\
 & & x_2 & - &x_3&=&4
\end{array}
\right.$\\
Augmented Matrix:
$\left[ \begin{array}{ccc|c}
2&3&-4&5\\
1&0&2&1\\
0&1&-1&4
\end{array}
\right]$\\
Vector Form:
$\left[ \begin{array}{c} 2\\1\\0 \end{array} \right]x_1+
\left[ \begin{array}{c} 3\\0\\1 \end{array} \right]x_2+
\left[ \begin{array}{c} -4\\2\\-1 \end{array} \right]x_3=
\left[ \begin{array}{c} 5\\1\\4 \end{array} \right]$
}\end{expl}
\begin{defn}\-\\
\textup{
If $\vec{v_1},\vec{v_2},...,\vec{v_p}$ are vectors in $R^N$ then the set of all linear combonations of $\vec{v_1},\vec{v_2},...,\vec{v_p}$ is denoted by Span\{$\vec{v_1},\vec{v_2},...,\vec{v_p}$\} and is called a subset of $R^N$ spanned (or generated) by $\vec{v_1},\vec{v_2},...,\vec{v_p}$.
}\end{defn}
\begin{expl}\-\\
\textup{
for $R^3$, describe Span\{$\left[\begin{array}{c}1\\0\\0\end{array}\right]$,$\left[\begin{array}{c}0\\0\\1\end{array}\right]$\}\\
All Linear Combontations we get the $x_1x_3$-plane\\
}\end{expl}
Remark: A system of linear equations is consistant if $\vec{b}$ is in Span\{$\vec{a_1},\vec{a_2},...,\vec{a_n}$ \}\\
\begin{expl}\-\\
\textup{
Determine if $\vec{b} = \left[\begin{array}{c}11\\-5\\9\end{array}\right]$ is in the Span\{$\left[\begin{array}{c}1\\0\\1\end{array}\right],
\left[\begin{array}{c}-2\\1\\2\end{array}\right],
\left[\begin{array}{c}-6\\7\\5\end{array}\right]$ \}\\
$\left[\begin{array}{c}1\\0\\1\end{array}\right]x_1+
\left[\begin{array}{c}-2\\1\\2\end{array}\right]x_2+
\left[\begin{array}{c}-6\\7\\5\end{array}\right]x_3=
\left[\begin{array}{c}11\\-5\\9\end{array}\right]$\\
$\left[\begin{array}{ccc|c}1&-2&-6&11\\0&1&7&-5\\1&2&5&9\end{array}\right]RowOperations\rightarrow\left[\begin{array}{ccc|c}1&-2&-6&11\\0&1&7&-5\\0&0&-17&-18\end{array}\right]$\\
Yes it is in Span because it is consistent!
}\end{expl}
Remark:\\
1) If the question is determine wheter the system is consistent or not. Then usually it is enought to get Echelon Form of the Augmented Matrix.\\
2) If the question is to solve the system, then we need Reduced Echelon Form of the Augmented Matrix
\begin{expl}\-\\
\textup{
$\begin{array}{ccc}
x_1+x_2-2x_3&=&5\\
x_1-x_2+x_3&=&7\\
5x_1-x_2-x_3&=&31
\end{array}=
\left[\begin{array}{ccc|c}1&1&-2&5\\1&-1&1&7\\5&-1&-1&31\end{array}\right]Row Operations \rightarrow\\
\left[\begin{array}{ccc|c}1&1&-2&5\\0&1&-\frac{3}{2}&-1\\0&0&0&0\end{array}\right]
\Rightarrow
\begin{array}{ccc}
x_1&=&-x_2+2x_3+5\\
x_2&=&\frac{3}{2}x_3-1\\
x_3&=&Parameter
\end{array}\\ \textup{Wrong Because $-x_2$ is not a parameter. If it's a pivot column, it can't be a parameter.}\\RowOperations\rightarrow
\left[\begin{array}{ccc|c}1&0&-\frac{1}{2}&6\\0&1&-\frac{3}{2}&-1\\0&0&0&0\end{array}\right]\Rightarrow
\begin{array}{ccc}
x_1&=&\frac{1}{2}x_3+6\\
x_2&=&\frac{3}{2}x_3-1\\
x_3&=&Parameter
\end{array}$\\
Remember: Echelon Form of a matrix is not unique. Reduced Echelon Form IS unique.
}\end{expl}
\section{The Matrix Equation $A\vec{x}=\vec{b}$}
Standard Form: 
$\begin{array}{ccccccccc}
a_{11}x_1&+&a_{12}x_2&+&...&+&a_{1n}x_n &=&b_1\\
a_{21}x_1&+&a_{22}x_2&+&...&+&a_{2n}x_n &=&b_2\\
\vdots&&\vdots&&&&\vdots&&\vdots\\
a_{m1}x_1&+&a_{m2}x_2&+&...&+&a_{mn}x_n&=& b_m\\
\end{array}$\\
\-\\ Matrix Form: 
$\underbrace{\left[ \begin{array}{cccc}
a_{11} & a_{12} & \ldots & a_{1n}\\
a_{21} & a_{22} & \ldots & a_{2n}\\
\vdots &\vdots&\ddots& \vdots\\
a_{m1} & a_{m2} & \ldots & a_{mn}\\
\end{array} \right]}_{A}
\underbrace{\left[\begin{array}{c}x_1\\x_2\\\vdots\\x_m\end{array}\right]}_{\vec{x}}=
\underbrace{\left[\begin{array}{c}b_1\\b_2\\\vdots\\b_m\end{array}\right]}_{\vec{b}}$\\
Theorem: The system $A\vec{x}=\vec{b}$ has a solution Iff $\vec{b}$ is a linear combination of $A, \vec{b} \in$ Span\{column vectors of $A$\}

\begin{expl}
\textup{
$A = \left[\begin{array}{ccc}3&5&-1\\2&0&4\\0&1&2\end{array}\right] \vec{b}=\left[\begin{array}{c}4\\2\\-1\end{array}\right]$\\
Standard Form: $\left\{\begin{array}{ccccccc}
3x_1&+&5x_2&-&1x_3&=&4\\
2x_1&&&+&4x_3&=&2\\
&&1x_2&+&2x_3&=&-1
\end{array}
 \right.$\\
Matrix Form: $\left[\begin{array}{ccc}3&4&-1\\2&0&4\\0&1&2\end{array} \right]\left[\begin{array}{c}x_1\\x_2\\x_3\end{array} \right] = \left[\begin{array}{c}4\\2\\-1\end{array} \right]$\\
Vector Form: $\left[\begin{array}{c}3\\2\\0\end{array}\right]x_1+\left[\begin{array}{c}5\\0\\1\end{array}\right]x_2+\left[\begin{array}{c}-1\\4\\2\end{array}\right]x_3=\left[\begin{array}{c}4\\2\\-1\end{array}\right]$
}\end{expl}
\begin{expl}
\textup{
How many rows have pivot positions?\\
$A = \left[\begin{array}{cccc}1&3&-2&-2\\0&1&-1&5\\-1&-2&1&7\\1&1&0&-6\end{array}\right]$
$Row Operations \rightarrow \left[\begin{array}{cccc}\circled{1}&3&-2&-2\\0&\circled{1}&-1&5\\0&0&0&\circled{6}\\0&0&0&0\end{array}\right]$\\
A as above\\ $A\vec{x}=\vec{b}$ Assume system is consistent\\
$Q_1$: On how many parameters does the solution depend?\\
Answer: One ($x_3$)\\
$Q_2$: Is it true that $A\vec{x}=\vec{b}$ has a solution for any $\vec{b} \in R^4$?\\
Answer: Only if there is a pivot position in each row. - So it's False.
}\end{expl}
\begin{expl}
\textup{
Do the vectors $\vec{v_1}=\left[\begin{array}{c}1\\3\\4\\-1\end{array}\right] \vec{v_2}=\left[\begin{array}{c}0\\7\\5\\-1\end{array}\right] \vec{v_3}=\left[\begin{array}{c}-1\\4\\2\\1\end{array}\right]$ Span $R^4$?\\
Only 3 vecrors, need at least 4 vectors to span $R^4$ (Still it is not enough, in general)\\
}\end{expl}
Theorem: Let $A$ be an m row by n column matrix then the following statements are equivilent.\\
a) For each $\vec{b}$ in $R^m$, the system $A\vec{x_1}=\vec{b}$ has a solution.\\
b) The columns of $A$ span $R^m$.\\
c) $A$ has a pivot position in every row.
\section{Section 1.5}
\section{Section 1.6}
\section{Section 1.7}
\section{Introduction to Linear Transformations}
\begin{expl}\-\\
\textup{
$A = \left[\begin{array}{cc}1&0\\0&-1\end{array}\right]$\\
$T(\vec{x})=A\vec{x}=\left[\begin{array}{cc}1&0\\0&-1\end{array}\right]\left[\begin{array}{c}x_1\\x_2\end{array}\right]=\left[\begin{array}{c}x_1\\-x_2\end{array}\right]$\\
}\end{expl}
\begin{defn}
\textup{
Transformation\\
A transformation (or function or mapping) $T$ from $R^n to R^m$ is a rule that assigns to each vector $\vec{x}$ in $R^n$ a vector $T(\vec{x})$ in $R^m$.\\
The set $R^n$ is called the \underline{Domain of $T$}\\
The set $R^m$ is called the \underline{Co-Domain of $T$}\\
$T(\vec{x})$ is called the \underline{image of $\vec{x}$} (under $T$)\\
$\{T(\vec{x}),\vec{x}\in R^n\}$ is called the \underline{range of $T$} (the set of all images)\\
}\end{defn}

\begin{defn}
\textup{
Linear Transformation\\
A transformation $T: R^n \rightarrow R^m$ is linear if:\\
(i) $T(\vec{u}+\vec{v})=T(\vec{u})+T(\vec{v})$ for any $\vec{u},\vec{v}$ from the domain of $T$\\
(ii) $T(c\vec{u})=cT(\vec{u})$ for any $\vec{u}$ and any scaler $c\in R$\\
}\end{defn}
\begin{expl} 
\textup{
.
}\end{expl}
\end{document}