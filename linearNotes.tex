\documentclass[a4paper,12pt,openany]{book}
\usepackage{fullpage}
\usepackage{setspace}
\usepackage{amsthm}
\usepackage{amssymb}
\usepackage{tikz}
\usepackage{mathtools}
\usepackage{hyperref}
\usepackage{multicol}
\newcommand*{\bfrac}[2]{\genfrac{}{}{0pt}{}{#1}{#2}}
\newcommand*\circled[1]{\tikz[baseline=(char.base)]{
            \node[shape=circle,draw,inner sep=2pt] (char) {#1};}}
\pagestyle{empty}
\theoremstyle{defn}
\newtheorem{defn}{Definition}[section]
\theoremstyle{expl}
\newtheorem{expl}{Example}[section]
\title{Math 208\\Linear Algebra\\Lecture Notes}
\date{Fall Semester 2013}
\author{Transcribed by R Barry Preston\\Based on Lecture by Dr.~Roman Dwilewicz\\Generated in \LaTeX{}}
\begin{document}
\maketitle
\tableofcontents 
\chapter{Linear Equations in Linear Algebra}
\section{Systems of Linear Equations}

\begin{defn}
	\textup{A Linear Equations is the variables $x_1,x_2...x_n$ is an equation that can be written in the form $a_1x_1+a_2x_2+...+a_nx_n = b$ where $a_1,a_2, ..., a_n$ are real coefficient and $b$ is a real number(and known)}
\end{defn}
\begin{defn}
	\textup{A System of Linear Equations
	$\left\{\begin{array}{c}
		a_{11}x_1+a_{12}x_2+...+a_{1n}x_n = b_1\\
		a_{21}x_1+a_{22}x_2+...+a_{2n}x_n = b_2\\
		\vdots\\
		a_{m1}x_1+a_{m2}x_2+...+a_{mn}x_n = b_m
	\end{array} \right.$
	m number of equations, n number of unknowns (standard form) (first index row number, second index col number)}\end{defn}
\begin{defn}\textup{A solution of the system is a list ($s_1,s_2,...,s_n$) of numbers that makes each equation a true statment when the values are substituted for $x_1,x_2,...,x_n$}\end{defn}
\begin{defn}\textup{Solution Set is the set of all possible solutions}\end{defn}

Geometric Interpretations
Example) Find the Solution set of the system \\
(a) $\left\{ \begin{array}{rcc}
x_1 - x_2 & = & 5 \\
2x_1 + x_2 & = & 7
\end{array} \right.
E_2-2E_1
\left\{ \begin{array}{rcc}
x_1 - x_2 & = & 5 \\
3x_2 & = & -3
\end{array} \right.
\frac{1}{3}E_2
\left\{ \begin{array}{rcc}
x_1 - x_2 & = & 5 \\
x_2 & = & -1
\end{array} \right.\\
E_1+E_2
\left\{ \begin{array}{rcc}
x_1 & = & 4 \\
x_2 & = & -1
\end{array} \right.$\\
(b) $\left\{ \begin{array}{rcc}
x_1 - 2x_2&=& 4 \\
-2x_1 + 4x_2 &=& -8
\end{array} \right.
E_2+2E_1
\left\{ \begin{array}{rcc}
x_1 - 2x_2&=& 4 \\
0&=&0
\end{array} \right.
~\left\{ \begin{array}{rcc}
x_1&=&2x_2+4\\
x_2&=&Parameter
\end{array} \right.
$\\
(c) $\left\{ \begin{array}{rcc}
x_1+3x_2&=&1\\
2x_1+6x_2&=&5
\end{array} \right.
E_2-2E_1
\left\{ \begin{array}{rcc}
x_1+3x_2&=&1\\
0x_1+0x_2&=&3
\end{array} \right.
$~Contradiction $0\neq 3$ so the solution set is empty
\begin{defn}\textup{A linear system is consistant if it has either one solution or infinitely many solutions}\end{defn}
\begin{defn}\textup{Standard Form of a Sysyem}
$\begin{array}{c}
a_{11}x_1+a_{12}x_2+...+a_{1n}x_n = b_1\\
a_{21}x_1+a_{22}x_2+...+a_{2n}x_n = b_2\\
\vdots\\
a_{m1}x_1+a_{m2}x_2+...+a_{mn}x_n = b_m
\end{array}$
\end{defn}
\begin{defn}\textup{Matrix of Coefficients} 
$\left[ \begin{array}{cccc}
a_{11} & a_{12} & \ldots & a_{1n}\\
a_{21} & a_{22} & \ldots & a_{2n}\\
\vdots &\vdots&\ddots& \vdots\\
a_{m1} & a_{m2} & \ldots & a_{mn}\\
\end{array} \right]$ \end{defn}
\begin{defn}\textup{Augmented Matrix of the System}
$\left[ \begin{array}{cccc|c}
a_{11} & a_{12} & \ldots & a_{1n} & b_1\\
a_{21} & a_{22} & \ldots & a_{2n} & b_2\\
\vdots &\vdots&\ddots& \vdots &\vdots\\
a_{m1} & a_{m2} & \ldots & a_{mn} & b_m\\
\end{array}\right]$ \end{defn}
The basic strategy to solve a system of linear equations is to replace one system with an equivalent one (ie one with the same solution set, that is easier to solve)\\\-\\
The basic operations to solve a system of linear equations (row operations)
\begin{enumerate}
\item Replacement: Replace one equation(row) by the sum of itself and a multiple of another equation(row)
\item Interchange: Interchange two equations(rows)
\item Scaling: Multiply all the terms in an equation(row) by a non zero constant
\end{enumerate}
\begin{expl}
	\textup{Solve the system of eqations} $\left\{\begin{array}{rcc}x_1-2x_2+x_3&=&1\\-x_1+x_2-3x_3&=&4\end{array}\right.\\
\left[\begin{array}{ccc|c}1&-2&1&1\\-1&1&-3&4\end{array}\right] R_2+R_1 \left[\begin{array}{ccc|c}1&-2&1&1\\0&-1&-2&5\end{array}\right] \bfrac{-R_2}{R_1+2R_2} \left[\begin{array}{ccc|c}1&0&5&-9\\0&1&2&-5\end{array}\right]
\left\{\begin{array}{rcc}x_1&=&-5x_3-9\\x_2&=&-2x_3-5\\x_3&=&Parameter\end{array}\right.\\
$\textup{Solution is not unique}\end{expl}
\begin{defn}
\textup{We say the two \underline{matricies are row equivalant} if there is a sequence of elementary row operations that transforms one matrix into the other $A\sim B$}\end{defn}
Remark: If the augmented matrix of two linear systems are row equivalent, then the two systems have the same solution set.
\begin{expl}
\textup{Determine if the system $\left\{\begin{array}{rcc}x_1-2x_2+x_3&=&1\\2x_1+x_2-x_3&=&2\\-x_1-8x_2+5x_3&=&4\end{array} \right.$ is consistent}\\
$\left[\begin{array}{ccc|c}
1&-2&1&1\\
2&1&-1&2\\
-1&-8&5&4
\end{array}\right]\bfrac{R_2-2R_1}{R_3+R_1}
\left[\begin{array}{ccc|c}
1&-2&1&1\\
0&5&-3&0\\
0&-10&6&5
\end{array}\right]R_3+2R_2
\left[\begin{array}{ccc|c}
1&-2&1&1\\
0&5&-3&0\\
0&0&0&5
\end{array}\right]\\
\left\{\begin{array}{rcl}
x_1-2x_2+x_3&=&1\\
5x_2-3x_3&=&0\\
0&=&5\rightarrow\textup{Contradiction}
\end{array}\right.\textup{The system is inconsistnent}
$\end{expl} 
\section{Row Reduction and Echelon Forms}
\begin{defn}\textup{A leading  of a row in a matrix is the left most non-zero entry} \end{defn}
Example) $\left[ \begin{array}{cccccc} 0 & 0 & \circled{7} & 3 & 4 & 1\\ \circled{2} & 4 & 0 &0 &0 &0 \\ 0 &0 &0 &0 &\circled{-2} &0 \end{array} \right] $

\begin{defn}\textup{A rectangular matrix is in \underline{echelon form} if it has the following three properties:
\begin{enumerate}
\item All non-zero rows are above any zero rows.
\item Each leading entry of a row is in a column to the right of the leading entry above it.
\item All entries in a column below a leading entry are zero.
\end{enumerate}} \end{defn}
\begin{defn}\textup{
	A matrix is in \underline{reduced echelon form} if it is in echelon form and addionally
	\begin{enumerate}
		\setcounter{enumi}{3}
		\item The leading entry in each non-zero row is 1
		\item Each leading 1 is the only non-zero entry in its column
	\end{enumerate}
}\end{defn}
\begin{expl}\textup{
$\blacksquare =$ non-zero, $* =$ any number\\
Echelon: $\left[\begin{array}{ccccc}
0&\blacksquare&*&*&*\\
0&0&0&\blacksquare&*\\
0&0&0&0&\blacksquare\\
0&0&0&0&0
\end{array}\right]$
Reduced Echelon: $\left[\begin{array}{ccccc}
0&1&*&0&0\\
0&0&0&1&0\\
0&0&0&0&1\\
0&0&0&0&0
\end{array}\right]$
}\end{expl}
\begin{expl}\textup{
Determine which matricies are in reduced echelon, echelon or neither\\
$\left[\begin{array}{cccc}
\circled{1}&1&1&0\\
0&0&0&\circled{1}\\
0&0&0&0
\end{array}\right]\rightarrow \textup{Reduced Echelon Form}\\
\left[\begin{array}{cccc}
\circled{1}&0&1&0\\
0&0&\circled{1}&0\\
0&0&0&\circled{1}
\end{array}\right]\rightarrow \textup{Echelon Form}\\
\left[\begin{array}{ccc}
\circled{1}&0&1\\
0&0&\circled{1}\\
0&\circled{1}&0
\end{array}\right]\rightarrow \textup{Neither}
$}\end{expl}
\begin{defn}\textup{\-\\
	A \underline{pivot position} in a matrix $A$ is a location in $A$ that corresponds to a leading entry in an echelon form of $A$\\
	A \underline{pivot column} is a column of $A$ that contains a pivot position  
}\end{defn}
\begin{expl}\textup{
Find the pivot positions in $A = \left[\begin{array}{cccc}0&0&2&3\\0&1&-1&-2\\0&1&1&1\end{array}\right]\\
\left[\begin{array}{cccc}0&0&2&3\\0&1&-1&-2\\0&1&1&1\end{array}\right] \bfrac{\bfrac{R_1\Leftrightarrow R_2}{then}}{\bfrac{}{R_2\Leftrightarrow R_3}}\left[\begin{array}{cccc}0&1&-1&-2\\0&1&1&1\\0&0&2&3\end{array}\right]R_2-R_1\left[\begin{array}{cccc}0&1&-1&-2\\0&0&2&3\\0&0&2&3\end{array}\right]\\R_3-R_2\left[\begin{array}{cccc}0&\circled{1}&-1&-2\\0&0&\circled{2}&3\\0&0&0&0\end{array}\right]\quad A = \left[\begin{array}{cccc}0&\circled{0}&2&3\\0&1&\circled{-1}&-2\\0&1&1&1\end{array}\right]
$}\end{expl}

\section{Vector Equations}
\begin{defn}
\textup{Vectors\\
In $R^2, \vec{v} = \left[\begin{array}{c} v_1 \\ v_2 \end{array}\right]$, in $R^3, \vec{v}=\left[\begin{array}{c} v_1 \\ v_2 \\v_3\end{array}\right]$, in $R^n, \vec{v} =\left[\begin{array}{c} v_1 \\ v_2 \\ \vdots \\ v_n\end{array}\right]$}\end{defn}
\begin{defn}
\textup{
Alebraic Operations of Vectors.\\
$\vec{u}=\left[\begin{array}{c} u_1\\u_2\\\vdots\\u_n \end{array}\right]$
$\vec{v}=\left[\begin{array}{c} v_1\\v_2\\\vdots\\v_n \end{array}\right]$\\
Addition: $\vec{u}+\vec{v}= \left[\begin{array}{c} u_1+v_1\\u_2+v_2\\\vdots\\u_n+v_n \end{array}\right]$\\
Multipy by Scaler: $c\in R\quad c\vec{v}=\left[\begin{array}{c} cv_1\\cv_2\\\vdots\\cv_n \end{array}\right]$
}\end{defn}

\begin{defn}
\textup{
Linear Combination of Vectors\\
$\vec{v_1},\vec{v_2},...,\vec{v_p}$ vectors in $R^n$\\
$c_1,c_2,...,c_n$ scalers\\
Linear Combination: $c_1\vec{v_1}+c_2\vec{v_2}+...+c_n\vec{v_n}$
}\end{defn}

\begin{defn}
\textup{
Vector Form of a System of Linear Equations\\
$\begin{array}{ccccccccc}
a_{11}x_1&+&a_{12}x_2&+&...&+&a_{1n}x_n &=&b_1\\
a_{21}x_1&+&a_{22}x_2&+&...&+&a_{2n}x_n &=&b_2\\
\vdots&&\vdots&&&&\vdots&&\vdots\\
a_{m1}x_1&+&a_{m2}x_2&+&...&+&a_{mn}x_n&=& b_m\\
\vec{a_1}&&\vec{a_2}&&&&\vec{a_n}&&\vec{b_n}\\
\end{array}$\\
Short Vector Form: $\vec{a_1}x_1+\vec{a_2}x_2+...+\vec{a_n}x_n=\vec{b_n}$\\
Long Vector Form: $\left[\begin{array}{c} a_{11}\\a_{21}\\\vdots\\a_{m1}\end{array}\right]x_1
+\left[\begin{array}{c} a_{12}\\a_{22}\\\vdots\\a_{m2}\end{array}\right]x_2+...
+\left[\begin{array}{c} a_{1n}\\a_{2n}\\\vdots\\a_{mn}\end{array}\right]x_n
=\left[\begin{array}{c} b_{1}\\b_{2}\\\vdots\\b_{m}\end{array}\right]$
}\end{defn}
\begin{expl}\-\\
\textup{
Standard Form: $\left\{ \begin{array}{ccccccc}
2x_1&+&3x_2&-&4x_3&=&5\\
x_1& & &+& 2x_3&=&1\\
 & & x_2 & - &x_3&=&4
\end{array}
\right.$\\
Augmented Matrix:
$\left[ \begin{array}{ccc|c}
2&3&-4&5\\
1&0&2&1\\
0&1&-1&4
\end{array}
\right]$\\
Vector Form:
$\left[ \begin{array}{c} 2\\1\\0 \end{array} \right]x_1+
\left[ \begin{array}{c} 3\\0\\1 \end{array} \right]x_2+
\left[ \begin{array}{c} -4\\2\\-1 \end{array} \right]x_3=
\left[ \begin{array}{c} 5\\1\\4 \end{array} \right]$
}\end{expl}
\begin{defn}\-\\
\textup{
If $\vec{v_1},\vec{v_2},...,\vec{v_p}$ are vectors in $R^N$ then the set of all linear combonations of $\vec{v_1},\vec{v_2},...,\vec{v_p}$ is denoted by Span\{$\vec{v_1},\vec{v_2},...,\vec{v_p}$\} and is called a subset of $R^N$ spanned (or generated) by $\vec{v_1},\vec{v_2},...,\vec{v_p}$.
}\end{defn}
\begin{expl}\-\\
\textup{
for $R^3$, describe Span\{$\left[\begin{array}{c}1\\0\\0\end{array}\right]$,$\left[\begin{array}{c}0\\0\\1\end{array}\right]$\}\\
All Linear Combontations we get the $x_1x_3$-plane\\
}\end{expl}
Remark: A system of linear equations is consistant if $\vec{b}$ is in Span\{$\vec{a_1},\vec{a_2},...,\vec{a_n}$ \}\\
\begin{expl}\-\\
\textup{
Determine if $\vec{b} = \left[\begin{array}{c}11\\-5\\9\end{array}\right]$ is in the Span\{$\left[\begin{array}{c}1\\0\\1\end{array}\right],
\left[\begin{array}{c}-2\\1\\2\end{array}\right],
\left[\begin{array}{c}-6\\7\\5\end{array}\right]$ \}\\
$\left[\begin{array}{c}1\\0\\1\end{array}\right]x_1+
\left[\begin{array}{c}-2\\1\\2\end{array}\right]x_2+
\left[\begin{array}{c}-6\\7\\5\end{array}\right]x_3=
\left[\begin{array}{c}11\\-5\\9\end{array}\right]$\\
$\left[\begin{array}{ccc|c}1&-2&-6&11\\0&1&7&-5\\1&2&5&9\end{array}\right]RowOperations\rightarrow\left[\begin{array}{ccc|c}1&-2&-6&11\\0&1&7&-5\\0&0&-17&-18\end{array}\right]$\\
Yes it is in Span because it is consistent!
}\end{expl}
Remark:\\
1) If the question is determine whether the system is consistent or not. Then usually it is enough to get Echelon Form of the Augmented Matrix.\\
2) If the question is to solve the system, then we need Reduced Echelon Form of the Augmented Matrix
\begin{expl}\-\\
\textup{
$\begin{array}{ccc}
x_1+x_2-2x_3&=&5\\
x_1-x_2+x_3&=&7\\
5x_1-x_2-x_3&=&31
\end{array}=
\left[\begin{array}{ccc|c}1&1&-2&5\\1&-1&1&7\\5&-1&-1&31\end{array}\right]Row Operations \rightarrow\\
\left[\begin{array}{ccc|c}1&1&-2&5\\0&1&-\frac{3}{2}&-1\\0&0&0&0\end{array}\right]
\Rightarrow
\begin{array}{ccc}
x_1&=&-x_2+2x_3+5\\
x_2&=&\frac{3}{2}x_3-1\\
x_3&=&Parameter
\end{array}\\ \textup{Wrong Because $-x_2$ is not a parameter. If it's a pivot column, it can't be a parameter.}\\RowOperations\rightarrow
\left[\begin{array}{ccc|c}1&0&-\frac{1}{2}&6\\0&1&-\frac{3}{2}&-1\\0&0&0&0\end{array}\right]\Rightarrow
\begin{array}{ccc}
x_1&=&\frac{1}{2}x_3+6\\
x_2&=&\frac{3}{2}x_3-1\\
x_3&=&Parameter
\end{array}$\\
Remember: Echelon Form of a matrix is not unique. Reduced Echelon Form IS unique.
}\end{expl}
\section[The Matrix Equation Ax=b]{The Matrix Equation $A\vec{x}=\vec{b}$}
Standard Form: 
$\begin{array}{ccccccccc}
a_{11}x_1&+&a_{12}x_2&+&...&+&a_{1n}x_n &=&b_1\\
a_{21}x_1&+&a_{22}x_2&+&...&+&a_{2n}x_n &=&b_2\\
\vdots&&\vdots&&&&\vdots&&\vdots\\
a_{m1}x_1&+&a_{m2}x_2&+&...&+&a_{mn}x_n&=& b_m\\
\end{array}$\\
\-\\ Matrix Form: 
$\underbrace{\left[ \begin{array}{cccc}
a_{11} & a_{12} & \ldots & a_{1n}\\
a_{21} & a_{22} & \ldots & a_{2n}\\
\vdots &\vdots&\ddots& \vdots\\
a_{m1} & a_{m2} & \ldots & a_{mn}\\
\end{array} \right]}_{A}
\underbrace{\left[\begin{array}{c}x_1\\x_2\\\vdots\\x_m\end{array}\right]}_{\vec{x}}=
\underbrace{\left[\begin{array}{c}b_1\\b_2\\\vdots\\b_m\end{array}\right]}_{\vec{b}}$\\
Theorem: The system $A\vec{x}=\vec{b}$ has a solution Iff $\vec{b}$ is a linear combination of $A, \vec{b} \in$ Span\{column vectors of $A$\}

\begin{expl}
\textup{
$A = \left[\begin{array}{ccc}3&5&-1\\2&0&4\\0&1&2\end{array}\right] \vec{b}=\left[\begin{array}{c}4\\2\\-1\end{array}\right]$\\
Standard Form: $\left\{\begin{array}{ccccccc}
3x_1&+&5x_2&-&1x_3&=&4\\
2x_1&&&+&4x_3&=&2\\
&&1x_2&+&2x_3&=&-1
\end{array}
 \right.$\\
Matrix Form: $\left[\begin{array}{ccc}3&4&-1\\2&0&4\\0&1&2\end{array} \right]\left[\begin{array}{c}x_1\\x_2\\x_3\end{array} \right] = \left[\begin{array}{c}4\\2\\-1\end{array} \right]$\\
Vector Form: $\left[\begin{array}{c}3\\2\\0\end{array}\right]x_1+\left[\begin{array}{c}5\\0\\1\end{array}\right]x_2+\left[\begin{array}{c}-1\\4\\2\end{array}\right]x_3=\left[\begin{array}{c}4\\2\\-1\end{array}\right]$
}\end{expl}
\begin{expl}
\textup{
How many rows have pivot positions?\\
$A = \left[\begin{array}{cccc}1&3&-2&-2\\0&1&-1&5\\-1&-2&1&7\\1&1&0&-6\end{array}\right]$
$Row Operations \rightarrow \left[\begin{array}{cccc}\circled{1}&3&-2&-2\\0&\circled{1}&-1&5\\0&0&0&\circled{6}\\0&0&0&0\end{array}\right]$\\
A as above\\ $A\vec{x}=\vec{b}$ Assume system is consistent\\
$Q_1$: On how many parameters does the solution depend?\\
Answer: One ($x_3$)\\
$Q_2$: Is it true that $A\vec{x}=\vec{b}$ has a solution for any $\vec{b} \in R^4$?\\
Answer: Only if there is a pivot position in each row. - So it's False.
}\end{expl}
\begin{expl}\textup{
Do the vectors $\vec{v_1}=\left[\begin{array}{c}1\\3\\4\\-1\end{array}\right] \vec{v_2}=\left[\begin{array}{c}0\\7\\5\\-1\end{array}\right] \vec{v_3}=\left[\begin{array}{c}-1\\4\\2\\1\end{array}\right]$ Span $R^4$?\\
Only 3 vecrors, need at least 4 vectors to span $R^4$ (Still it is not enough, in general)
}\end{expl}
\begin{defn}\textup{Theorem: Let $A$ be an m row by n column matrix then the following statements are equivilent.\\
a) For each $\vec{b}$ in $R^m$, the system $A\vec{x_1}=\vec{b}$ has a solution.\\
b) The columns of $A$ span $R^m$.\\
c) $A$ has a pivot position in every row.}\end{defn}
\begin{expl}
\textup{
Do the columns of $A=\left[\begin{array}{cccc}1&-1&5&0\\2&0&4&2\\4&1&5&5\end{array}\right]$ span $R^3$?\\
$A\quad\widetilde{\bfrac{R_2-2R_1}{R_3-4R_1}}\quad\left[\begin{array}{cccc}1&-1&5&0\\0&2&-6&2\\0&5&-15&5\end{array}\right]
\widetilde{R_3-\frac{5}{2}R_2}\left[\begin{array}{cccc}\circled{1}&-1&5&0\\0&\circled{2}&-6&2\\0&0&0&0\end{array}\right]$\\
NO, the columns of $A$ do NOT span $R^3$ because all the vectors lie in a plane(no z component)
}\end{expl}
\begin{defn}\textup{
Notation of Matrices\\
Identity Matrix(Square): $I_n = \left[\begin{array}{cccc}1&0&...&0\\0&1&...&0\\\vdots&\vdots&\ddots&\vdots\\0&0&...&1\end{array}\right],I_1 = \left[\begin{array}{c}1\end{array}\right], I_2 = \left[\begin{array}{cc}1&0\\0&1\end{array}\right]$\\
Zero Matrix(Rectangle): $0 = \left[\begin{array}{ccc}0&...&0\\\vdots&\ddots&\vdots\\0&...&0\end{array}\right]$
}\end{defn}
\section{Solution Sets of Linear Systems}
$A\vec{x}=\vec{b}, A=m\times n, \vec{x} = n\times 1, \vec{b} = m\times 1\\
1^{\textup{st}} \textup{Case}: \vec{b}=\vec{0}, A\vec{x}=\vec{0}, \textup{Homogeneous System}\\
\textup{Homogeneous system is always consistant}\\
2^{\textup{nd}} \textup{Case}: \vec{b}\neq\vec{0}, A\vec{x}=\vec{b}, \textup{Nonhomogeneous System}\\
\textup{Nonhomogeneous system can be either consistant or inconsistant}$
\section{Applications of Linear Systems}
\section{Linear Independence}
\section{Introduction to Linear Transformations}
\begin{expl}\-\\
\textup{
$A = \left[\begin{array}{cc}1&0\\0&-1\end{array}\right]$\\
$T(\vec{x})=A\vec{x}=\left[\begin{array}{cc}1&0\\0&-1\end{array}\right]\left[\begin{array}{c}x_1\\x_2\end{array}\right]=\left[\begin{array}{c}x_1\\-x_2\end{array}\right]$\\
}\end{expl}
\begin{defn}
\textup{
Transformation\\
A transformation (or function or mapping) $T$ from $R^n to R^m$ is a rule that assigns to each vector $\vec{x}$ in $R^n$ a vector $T(\vec{x})$ in $R^m$.\\
The set $R^n$ is called the \underline{Domain of $T$}\\
The set $R^m$ is called the \underline{Co-Domain of $T$}\\
$T(\vec{x})$ is called the \underline{image of $\vec{x}$} (under $T$)\\
$\{T(\vec{x}),\vec{x}\in R^n\}$ is called the \underline{range of $T$} (the set of all images)\\
}
\end{defn}

\begin{defn}
\textup{
Linear Transformation\\
A transformation $T: R^n \rightarrow R^m$ is linear if:\\
(i) $T(\vec{u}+\vec{v})=T(\vec{u})+T(\vec{v})$ for any $\vec{u},\vec{v}$ from the domain of $T$\\
(ii) $T(c\vec{u})=cT(\vec{u})$ for any $\vec{u}$ and any scaler $c\in R$\\
}\end{defn}
\begin{expl} 
\textup{
Example of Lienar Transformations\\
Matrix Transformation: $A, m~x~n \\ T:R^n\rightarrow R^m, T(\vec{x})=A\vec{x}$\\
(i)$T(\vec{u}+\vec{v})=A(\vec{u}+\vec{v})=A\vec{u}+A\vec{v}=T(\vec{u})+T(\vec{v})$\\
(ii)$T(c\vec{u})=Ac\vec{u}=cA\vec{u}=cT(\vec{u})$\\
For Linear Transformation if the two conditions from the definition of linear transformation are satisfied $\rightarrow A$ matrix transformations is a linear transformations
}\end{expl}
\begin{expl}
\textup{
Check if $T:R^2\rightarrow R^2\quad T\left(\left[\begin{array}{c}x_1\\x_2\end{array}\right]\right)=\left[\begin{array}{c}x_1+x_2\\x_2\end{array}\right]$ is a linear transformation.\\
Two methods to prove the problme: (I) Check the conditions (i),(ii) directly, (II) Show that $T$ is a matrix transformation.\\((II) is easier)\\
(II) $T\left(\left[\begin{array}{c}x_1\\x_2\end{array}\right]\right)=\left[\begin{array}{cc}1&1\\0&1\end{array}\right]\left[\begin{array}{c}x_1\\x_2\end{array}\right]=\left[\begin{array}{c}x_1+x_2\\x_2\end{array}\right]$\\
}\end{expl}

\section{Solution Sets of Linear Systems}
\chapter{Matrix Algebra}
\section{Matrix Operations}
$A = \left[\begin{array}{ccccc}
a_{11}&...&a_{1j}&...&a_{1n}\\
\vdots&\vdots&\vdots&\vdots&\vdots\\
a_{i1}&...&a_{ij}&...&a_{in}\\
\vdots&\vdots&\vdots&\vdots&\vdots\\
a_{m1}&...&a_{mj}&...&a_{mn}
\end{array}\right]
\textup{Diagonal Matrix} = \left[\begin{array}{cccc}
a_{11}&0&...&0\\
0&a_{22}&...&0\\
\vdots&\vdots&\ddots&\vdots\\
0&0&...&a_{nn}
\end{array}\right]\\
\textup{Identity Matrix(Square):} I_n = \left[\begin{array}{cccc}1&0&...&0\\0&1&...&0\\\vdots&\vdots&\ddots&\vdots\\0&0&...&1\end{array}\right] \textup{Zero Matrix(Rectangle):} 0 = \left[\begin{array}{ccc}0&...&0\\\vdots&\ddots&\vdots\\0&...&0\end{array}\right]$\\
Basic Operations:\\
Addition(+) Subtraction(-) Matrices should be the same size\\
\begin{multicols}{2}
\begin{enumerate}
\item $A+B=B+A$
\item $(A+B)+C=A+(B+C)$
\item $A+0=A$
\item $r(A+B)=rA+rB$
\item $(r+s)A=rA+sA$
\item $(rs)A=r(sA)$
\end{enumerate}
\end{multicols}
\noindent Matrix Multiplication\\
$A_{m\times n} \times B_{n\times p} = AB_{m\times p}$ order of multiplication IS important ($AB \neq BA$).\\
$\left[\begin{array}{cccc}&&&\\r&o&w&i\\&&&\end{array}\right]\left[\begin{array}{ccc}&c&\\&o&\\&l&\\&j&\end{array}\right]=\left[\begin{array}{ccc}&&\\&c_{ij}&\\&&\end{array}\right]$\\
\begin{expl}
\textup{Compute AB where $A = \left[\begin{array}{cc}3&2\\0&1\\1&-2\end{array}\right], B = \left[\begin{array}{cc}2&-1\\0&5\end{array}\right]$}\\
$AB=\left[\begin{array}{cc}3&2\\0&1\\1&-2\end{array}\right]\left[\begin{array}{cc}2&-1\\0&5\end{array}\right]=\left[\begin{array}{cc}6&7\\0&4\\2&-11\end{array}\right]$
\end{expl}
\begin{expl}
\textup{Find the entries in the $2^{\textup{nd}}$ column in $AB$ where\\ $A=\left[\begin{array}{ccc}1&2&-1\\2&0&1\\0&1&3\end{array}\right],B=\left[\begin{array}{cccc}4&1&0&-1\\2&0&1&-2\\5&1&0&4\end{array}\right]$}\\
$\left[\begin{array}{ccc}1&2&-1\\2&0&1\\0&1&3\end{array}\right]\left[\begin{array}{cccc}4&1&0&-1\\2&0&1&-2\\5&1&0&4\end{array}\right]=\left[\begin{array}{cccc}x&0&x&x\\x&3&x&x\\x&3&x&x\end{array}\right]$
\end{expl}
\noindent Properties of Matrix Multiplication\\
Let $A$ be $m\times n$ and let $B$ and $C$ have sizes for which the operations are defined.
\begin{multicols}{2}
\begin{enumerate}
\item $A(BC)=(AB)C$
\item $A(B+C)=AB+AC$
\item $(B+C)A=BA+CA$
\item $r(AB)=(rA)B=A(rB)$
\item $I_mA=A=AI_n$
\end{enumerate}
\end{multicols}
\noindent WARNING!
\begin{enumerate}
\item In general $AB\neq BA$ (if $AB=BA$, we say the matrices commute)
\item The cancellation laws do not hold, in general $AB=AC \nRightarrow B=C$ (in general)
\item If $AB=0$, then in general neither $A=0$ or $B=0$
\end{enumerate}
\begin{expl}\textup{Compute $AB$ and $BA$ where $A=\left[\begin{array}{cc}1&-1\\-1&1\end{array}\right],B=\left[\begin{array}{cc}2&4\\2&4\end{array}\right]$}\\
$AB = \left[\begin{array}{cc}1&-1\\-1&1\end{array}\right]\left[\begin{array}{cc}2&4\\2&4\end{array}\right] = \left[\begin{array}{cc}(1*2)+(-1*2)&(-1*4)+(1*4)\\(-1*2)+(1*2)&(-1*4)+(1*4)\end{array}\right]=\left[\begin{array}{cc}0&0\\0&0\end{array}\right]\\
BA = \left[\begin{array}{cc}2&4\\2&4\end{array}\right]\left[\begin{array}{cc}1&-1\\-1&1\end{array}\right] = \left[\begin{array}{cc}(2*1)+(4*-1)&(2*-1)+(4*1)\\(2*1)+(4*-1)&(2*-1)+(4*1)\end{array}\right]=\left[\begin{array}{cc}-2&2\\-2&2\end{array}\right]\\
AB=\left[\begin{array}{cc}0&0\\0&0\end{array}\right]\neq \left[\begin{array}{cc}-2&2\\-2&2\end{array}\right]=BA$\\
Note: if $AB=BA$, we can say that $A$ and $B$ commute with one another.
\end{expl}
\begin{defn}\textup{Powers of a Matrix\\
$A, n\times n$ matrix\\}
$A^2 = AA\quad A^3 = AAA\quad A^k=AAA...A (k \textup{times})(k=1,2,3...)\\
A^0=I$
\end{defn}
\begin{expl}\textup{Computer $A^{179}$ if $A=\left[\begin{array}{cc}0&0\\1&0\end{array}\right]$\\}
$A^2=\left[\begin{array}{cc}0&0\\1&0\end{array}\right]\left[\begin{array}{cc}0&0\\1&0\end{array}\right]=\left[\begin{array}{cc}0&0\\0&0\end{array}\right]\\
A^3=\left[\begin{array}{cc}0&0\\1&0\end{array}\right]\left[\begin{array}{cc}0&0\\1&0\end{array}\right]\left[\begin{array}{cc}0&0\\1&0\end{array}\right]=\left[\begin{array}{cc}0&0\\0&0\end{array}\right]\left[\begin{array}{cc}0&0\\1&0\end{array}\right]=\left[\begin{array}{cc}0&0\\0&0\end{array}\right]\\
A^k = \left[\begin{array}{cc}0&0\\0&0\end{array}\right] (k \geq 2) \quad A^{179}=\left[\begin{array}{cc}0&0\\0&0\end{array}\right]=0$
\end{expl}
\begin{defn}\textup{Transpose of a matrix\\
Let $A$ be a $m\times n$ matrix. The transpose of $A$ is $m\times n$ matrix denoted by $A^T$, whoes columns are formed from the corisponding rows of $A$\\}
\end{defn}
\begin{expl}
$A=\left[\begin{array}{ccc}1&2&5\\4&3&7\end{array}\right], A^T=\left[\begin{array}{cc}1&4\\2&3\\5&7\end{array}\right]$
\end{expl}
\begin{expl}\textup{Compute the transpose of the matricies\\}
$A=\left[\begin{array}{cc}a&b\\c&d\end{array}\right]\quad A^T = \left[\begin{array}{cc}a&c\\b&d\end{array}\right]\\
B=\left[\begin{array}{cc}5&8\\3&9\\1&0\end{array}\right]  \quad B^T=\left[\begin{array}{ccc}5&3&1\\8&9&0\end{array}\right] \\
C=\left[\begin{array}{crccc}1&-2&4&-7&9\\0&1&2&-5&3\end{array}\right]  \quad C^T=\left[\begin{array}{rr}1&0\\-2&1\\4&2\\-7&-5\\9&3\end{array}\right]$
\end{expl}
\noindent Properties of Transpose
\begin{multicols}{2}
\begin{enumerate}
\item $(A^T)^T=A$
\item $(A+B)^T=A^T=B^T$
\item $(r^A)^T=rA^T$
\item $(AB)^T=B^TA^T$
\end{enumerate}
\end{multicols}
\section{The Inverse of a Matrix}
\end{document}
